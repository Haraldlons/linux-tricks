\documentclass[11pt]{article}
\usepackage[norsk]{babel}	% Norwegian names on Introduction and other places
\usepackage[T1]{fontenc}		% Norwegian charset tegnsett (æøå)
\usepackage[utf8]{inputenc}	% Norwegian charset
\usepackage{geometry}		% Recommended package for controlling margins.

\usepackage{
			amsmath,
			caption,
			amssymb, %Comments here are fine
			float, 
			lmodern, 
			parskip, 
			textcomp,
			}
\usepackage{booktabs}
\usepackage{graphicx}
\usepackage{hyperref}
\hypersetup{
    colorlinks=true, 
    linkcolor=blue, % Color of links in 'innholdsfortegnelse'
    filecolor=blue, % Doesn't show when colorlinks=true. It's the border-color. 
    urlcolor=blue, % Links wil be nice blue color
}
\newcommand\tab[1][1cm]{\hspace*{#1}}

\usepackage{listings} % für Formatierung in Quelltexten
\lstset{
    escapeinside={(*@}{@*)},          % if you want to add LaTeX within your code
}

\begin{document}

\title{\textbf{LaTeX Article Template}}
\author{Author: Harald Lønsethagen}
% \date{\today}
\date{Last edited: 22.06.2017\tab Version: 1.0}

\maketitle
 
  %{\begin{figure}[H]
   % \centering
    %\includegraphics{Designimage}
%\end{figure}}

% \clearpage\thispagestyle{empty}
\makeatletter
%\renewcommand\tableofcontents{\@starttoc{toc}}
%\makeatother
\tableofcontents
\newpage

\section{Innledning}
Dette er et template-dokument som er ment som å brukes som utgangspunkt når man lager nye \LaTeX-dokumenter. Tenkt bruk er at man kopierer .tex-filen og sletter alt innholdet for så å lage sit eget dokument med eget innhold. \\
Det er lagt til mange eksempler på ulike funksjoner i \LaTeX \space slik at man fort kan bare kopiere/beholde .tex-kodebitene som man trenger. F.eks. å legge til et bilde med bildetekst. 

\section{Linker}
\href{www.vg.no}{Dette} er en link, som får en tydelig farge som viser at det er en link. Fargen er definert øverst i dokumentet.



\section{Seksjoner}
\subsection{Subseksjoner}
Slik ser en subseksjon ut og skrives
\begin{lstlisting}
	\subsection{}
\end{lstlisting}
Sjekk .tex filen for å se hvordan jeg \textit{escaper} latex-kommandoer, dvs at jeg faktisk ikke lager en ny subseksjon men skriver det ut så man kan lese det i PDF-filen.
\subsubsection{Subsubseksjoner}
Man kan lage subsubseksjoner også. Skriv: 
\begin{lstlisting}
	\subsubsection{}
\end{lstlisting}
Der er \textbf{ikke} mulig å lage subsubsubseksjoner.


\section{Tekstmodifisering}
Man kan endre fonten til tekst på enkle kommandoer.\newline{}
\begin{lstlisting}
	\textbf{Bold} (*@ $\rightarrow$ \textbf{Bold}@*) 
\end{lstlisting}
\begin{lstlisting}
	\textit{Italic} (*@ $\rightarrow$ \textit{Italic}@*) 
\end{lstlisting}
\begin{lstlisting}
	\textsc{Sc} (*@ $\rightarrow$ \textsc{Sc}@*) 
\end{lstlisting}
\begin{lstlisting}
	\underline{Underline} (*@ $\rightarrow$ \underline{Underline}@*) 
\end{lstlisting}
\begin{lstlisting}
	\emph{Emph} (*@ $\rightarrow$ \emph{emph}@*) 
\end{lstlisting}
\begin{lstlisting}
	No commands (*@ $\rightarrow$ No commands@*) 
\end{lstlisting}

\textbf{Ibsens ripsbærbusker og andre buskvekster, samt en djerv dvergbjerk}

\textit{Ibsens ripsbærbusker og andre buskvekster, samt en djerv dvergbjerk}

\underline{Ibsens ripsbærbusker og andre buskvekster, samt en djerv dvergbjerk}

\emph{Ibsens ripsbærbusker og andre buskvekster, samt en djerv dvergbjerk}


% \section{Figurer}
% \begin{figure}[H]
%     \centering
%     \includegraphics[scale=0.3]{roborace}
%     \caption{Dette kalles 'caption' og er bildeteksten til figuren}
%     \label{fig:com}
% \end{figure}

\section{Lister}
\subsection{Numererte liser}
\begin{enumerate}
	\item Dette er første 'item' i listen
	\item Man kan ha mange items
	\item Bare å ramse opp masse meningsløst
	\item Tips: Ikke ha punktum på slutten når man lister opp ting
\end{enumerate}

\subsection{Punktliste}
\begin{itemize}
	\item Dette er første \textit{item} i punktlisten
	\item Om det er ikke er noe grunn til å nummerere er det best å bruke punktliste
\end{itemize}

\subsection{Liste inni liste}
\begin{itemize}
	\item Det er fint mulig å ha en liste inni en liste
	\item Dette er fortsatt i den opprinnelige listen
	\begin{itemize}
		\item Dette er et item i listen listen
		\item Item 3.2
		\begin{itemize}
			\item Liste inni liste inni liste
			\begin{itemize}
				\item Det er \textbf{ikke} mulig å flere nivåer av lister enn dette.
			\end{itemize}
		\end{itemize}
	\end{itemize}
	\item Enda et item
	\begin{enumerate}
		\item Men det er mulig å lage en numerert liste inni en punktliste
		\begin{itemize}
			\item Og lage en punktliste inni en numerert liste som er inni en punktliste
		\end{itemize}
	\end{enumerate}
\end{itemize}

\section{Matematiske formler}
\LaTeX \space er fantastisk god til å vise matematiske formler og utregninger. Tar litt tid å bli vant med, men når man først lærer seg det ser det helt \textbf{porno} ut. 
$$\cos (2\theta) = \cos^2 \theta - \sin^2 \theta$$
Next is an \textit{equation:}
\begin{equation}
\cos (2\theta) = \cos^2 \theta - \sin^2 \theta
\end{equation}

\begin{equation}\lim_{x \to \infty} \exp(-x) = 0\end{equation}

\begin{equation}\lim_{x \to \infty} \exp(-x) = 0\end{equation}

\begin{equation} \frac{n!}{k!(n-k)!} = \binom{n}{k}\end{equation}

\begin{equation} \displaystyle\sum_{i=1}^{10} t_i\end{equation}

\begin{equation} \int_0^\infty \mathrm{e}^{-x}\,\mathrm{d}x
\end{equation}

Sjekk også \href{https://www.tug.org/texshowcase/#math}{dette} ut
% \begin{equation} 
% ( a ), [ b ], \{ c \}, | d |, \| e \|,
% \langle f \rangle, \lfloor g \rfloor,
% \lceil h \rceil, \ulcorner i \urcorner
% \end{equation}

$$
\sum, \prod,	\coprod, \bigoplus, \bigotimes,	\bigodot, \bigcup, \bigcap,\biguplus,\bigsqcup,\bigvee,\bigwedge,\int,\oint,\iint[3],\iiint[3],\iiiint[3],\iiiint,\idotsint[3]
$$

$$
( a ), [ b ], \{ c \}, | d |, \| e \|,
\langle f \rangle, \lfloor g \rfloor,
\lceil h \rceil, \ulcorner i \urcorner
$$




\section{Refferanser}
\begin{itemize}
\item \href{https://en.wikipedia.org/wiki/HTML}{https://en.wikipedia.org/wiki/HTML}
\item \href{https://en.wikipedia.org/wiki/JavaScript}{https://en.wikipedia.org/wiki/JavaScript}
\item \url{https://www.codecademy.com/learn/}
\item \href{http://getbootstrap.com/}{http://getbootstrap.com/}
\item \href{https://www.datatables.net/}{https://www.datatables.net/}
\end{itemize}
 

\end{document}
